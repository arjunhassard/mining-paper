\documentclass[longbibliography,nofootinbib]{revtex4-1}

\newcommand{\kms}{NuCypher KMS}

\usepackage{listings}
\usepackage{graphicx}
\usepackage{amsmath}
\usepackage[margin=5pt]{subfig}
\usepackage[usenames]{color}

\renewcommand{\baselinestretch}{1.4}
\setlength{\parskip}{1em}
\definecolor{darkgreen}{rgb}{0.00,0.50,0.25}
\definecolor{darkblue}{rgb}{0.00,0.00,0.67}
\newcommand{\figref}[1]{Fig.~\ref{#1}}
\usepackage[breaklinks,pdftitle={NuCypher KMS: Mining}, pdfauthor={Michael Egorov},colorlinks,urlcolor=blue,citecolor=darkgreen,linkcolor=darkblue]{hyperref}
\graphicspath{{pdf/}}

\usepackage[T1]{fontenc}
\usepackage{lmodern}
\lstset{
    basicstyle=\ttfamily,
    basewidth={0.5em, 0.5em},
    columns=fullflexible,
}

\begin{document}

\title{\kms: Mining}

\author{Michael Egorov}
\email{michael@nucypher.com}
\affiliation{NuCypher}

\begin{abstract}
    This paper describes mining mechanisms and economics in \kms.
    It includes inflation rates, mechanisms to incentivise long-term stakers
    and estimates of number of coins generated by nodes running in typical modes.
    Also, optimal strategies for stakers who may be affected by market volatility are proposed.
\end{abstract}

\date{\today}
\maketitle

\section{Introduction}

\end{document}
